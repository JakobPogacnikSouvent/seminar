\documentclass[a4paper,12pt]{article}

\usepackage[slovene]{babel}
\usepackage{amsfonts,amssymb,amsmath}
\usepackage{amsthm}
\usepackage[utf8]{inputenc}
\usepackage[T1]{fontenc}
\usepackage{lmodern}
\usepackage{graphicx}


\def\N{\mathbb{N}} % mnozica naravnih stevil
\def\Z{\mathbb{Z}} % mnozica celih stevil
\def\Q{\mathbb{Q}} % mnozica racionalnih stevil
\def\R{\mathbb{R}} % mnozica realnih stevil
\def\C{\mathbb{C}} % mnozica kompleksnih stevil


\newcommand{\cay}[2]{$Cay(#1, #2)$}
\newcommand{\aut}[2]{Aut(#1, #2)}
\newcommand{\cit}[1]{(Izrek \ref{#1})}

\def\qed{$\hfill\Box$}   % konec dokaza
\def\qedm{\qquad\Box}   % konec dokaza v matematičnem načinu
\newtheorem{izrek}{Izrek}
\newtheorem{trditev}{Trditev}
\newtheorem{posledica}{Posledica}
\newtheorem{lema}{Lema}
\newtheorem{opomba}{Opomba}
\newtheorem{definicija}{Definicija}
\newtheorem{zgled}{Zgled}

\title{Proste grupe in drevesa \\ 
\Large Seminar}
\author{Jakob Pogačnik Souvent \\
Fakulteta za matematiko in fiziko \\
Oddelek za matematiko}

\begin{document}


%%%%%%%%%%%%%%%%%%%%%%%%%%%%%%%%%%%%%%%%%%%%%%%%%%%%%%%%%%%%%%%%%%%%%


\maketitle


%%%%%%%%%%%%%%%%%%%%%%%%%%%%%%%%%%%%%%%%%%%%%%%%%%%%%%%%%%%%%%%%%%%%%

\section{Uvod}
Lorem
\section{Proste grupe}

\begin{definicija}[Prosta grupa]
    Naj bo $S$ podmnožica grupe $F$. Pravimo, da $S$ \textbf{prosto generira} $F$, če je $F$ generirana z $S$ in velja:

    Za vsako grupo $G$ in vsako preslikavo $\varphi : S \longrightarrow G$ obstaja enolično določen homomorfizem $\overline{\varphi} : F \longrightarrow G$ ki razširi $\varphi$.

    Grupi $F$ pravimo \textbf{prosta}, če vsebuje kakšno podmnožico, ki jo prosto generira, zgornji lastnosti pa pravimo \textbf{univerzalna lastnost} prostih grup.
\end{definicija}

TODO: Formalna definicija proste grupe nad množico in definicija z reduced words

\section{Cayleyjevi grafi}

\begin{definicija}
    \textbf{Graf} TODO
\end{definicija}

\begin{definicija}[Cayleyev graf]
    Naj bo $S$ podmnožica, ki generira grupo $G$. \textbf{Cayleyjev graf} $G$ glede na generator $S$ je graf \cay{G}{S} katerega množica vozlišč je množica $G$
    in katerega množica robov je množica
    $$
    \{\{g, g \cdot s\} | g \in G, s \in (S \cup S^{-1}) \setminus \{e\} \}.
    $$
\end{definicija}

Dve vozlišči v Cayleyjevem grafu sta si torej sosednji natanko tedaj, ko se razlikujeta le za desno množenje z elementom (ali inverzom) dane množice $S$, ki generira $G$.

\begin{izrek}
    \label{izr:freeCayleyTree}
    Naj bo $F$ grupa, prosto generirana z $S \subset F$. Potem je graf \cay{F}{S} drevo.
\end{izrek}

\begin{opomba}
    Inverz v splošnem ne drži. t.j. ni res da za vsak \cay{F}{S} sledi, da je $F$ prosto generirana z $S$. TODO: protiprimer
\end{opomba}

TODO: Če dodamo še predpostavko da $s \cdot t \neq e \forall s, t \in S$ inverz drži.


\section{Delovanje}

\begin{definicija}[Prosto delovanje na množici]
    Naj grupa $G$ deluje na množici $X$. Pravimo, da je delovanje \textbf{prosto}, če velja:
    $$
        g \cdot x \neq x
    $$
    za vsak $g \in G \setminus \{e\}$ in vsak $x \in X$.
\end{definicija}

\begin{definicija}[Prosto delovanje na grafu]
    Naj grupa $G$ deluje na grafu $(V, E)$. Označimo delovanje kot preslikavo $\rho : G \longrightarrow \aut{V}{E}$. To delovanje je \textbf{prosto}, če za vsak $g \in G\setminus\{e\}$ velja:
    \begin{align*}
        \forall v \in V : (\rho(g))(v) &\neq v \text{, in}\\
        \forall \{v, v'\} \in E : \{(\rho(g))(v), (\rho(g))(v')\} &\neq \{v, v'\}
    \end{align*}

\end{definicija}

\begin{izrek}
    Naj bo $G$ grupa in $S$ neka množica ki generira $G$. Potem je delovanje $G$ na \cay{G}{S} z levo translacijo
    $$
    g \cdot v = gv
    $$
    prosto natanko tedaj, ko $S$ ne vsebuje nobenega elementa reda $2$.
\end{izrek}

\begin{proof}
    Ker velja $g \cdot g' = gg'$ po definiciji delovanja in $gg' = g' \Longleftrightarrow g = e$ zaradi dejstva da sta $g$ in $g'$ elementa grupe $G$
    je delovanje $G$ na vozliščih \cay{G}{S} vedno prosto in ustreza prvemu delu definicija za prosto delovanje na grafu.
\end{proof}

\section{Delovanje prostih grup na drevesih}

\begin{izrek}
    Grupa je prosta natanko tedaj ko ima neko prosto delovanje na nepraznem drevesu.
\end{izrek}

\begin{proof}
    Naj bo $F$ prosta grupa prosto generirana z $S \subset F$ po izreku \ref{izr:freeCayleyTree} je njen Cayleyev graf \cay{F}{S} drevo.
    Oglejmo si delovanje $F$ na \cay{F}{S} z levo translacijo.
    $$
    f \cdot v = fv
    $$
\end{proof}
    
\section{Nielsen-Schreirerjev izrek}

\end{document}