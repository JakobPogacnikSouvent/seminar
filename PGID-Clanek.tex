\documentclass[a4paper,12pt]{article}

\usepackage[slovene]{babel}
\usepackage{amsfonts,amssymb,amsmath}
\usepackage{amsthm}
\usepackage[utf8]{inputenc}
\usepackage[T1]{fontenc}
\usepackage{lmodern}
\usepackage{graphicx}

\usepackage{tikz-cd}

\usetikzlibrary{babel}

\def\N{\mathbb{N}} % mnozica naravnih stevil
\def\Z{\mathbb{Z}} % mnozica celih stevil
\def\Q{\mathbb{Q}} % mnozica racionalnih stevil
\def\R{\mathbb{R}} % mnozica realnih stevil
\def\C{\mathbb{C}} % mnozica kompleksnih stevil


\newcommand{\cay}[2]{$Cay(#1, #2)$}
\newcommand{\aut}[2]{Aut(#1, #2)}
\newcommand{\cit}[1]{(Izrek \ref{#1})}
\newcommand{\St}{\widetilde{S}}
\newcommand{\As}{A^*}
\newcommand{\Freds}{F_{red}(S)}

\def\qed{$\hfill\Box$}   % konec dokaza
\def\qedm{\qquad\Box}   % konec dokaza v matematičnem načinu
\newtheorem{izrek}{Izrek}
\newtheorem{trditev}{Trditev}
\newtheorem{posledica}{Posledica}
\newtheorem{lema}{Lema}
\newtheorem{opomba}{Opomba}
\newtheorem{definicija}{Definicija}
\newtheorem{zgled}{Zgled}

\title{Proste grupe in drevesa \\ 
\Large Seminar}
\author{Jakob Pogačnik Souvent \\
Fakulteta za matematiko in fiziko \\
Oddelek za matematiko}

\begin{document}


%%%%%%%%%%%%%%%%%%%%%%%%%%%%%%%%%%%%%%%%%%%%%%%%%%%%%%%%%%%%%%%%%%%%%


\maketitle


%%%%%%%%%%%%%%%%%%%%%%%%%%%%%%%%%%%%%%%%%%%%%%%%%%%%%%%%%%%%%%%%%%%%%

\section{Uvod}
TODO
TODO: primeri
\section{Proste grupe}

\begin{definicija}[Prosta grupa]
    Naj bo $S$ množica. Za grupo $F$, ki vsebuje $S$ Pravimo, da $S$ \textbf{prosto generira} $F$, če velja:

    Za vsako grupo $G$ in vsako preslikavo $\varphi : S \longrightarrow G$ obstaja enolično določen homomorfizem $\overline{\varphi} : F \longrightarrow G$ ki razširi $\varphi$. %(torej $\overline{\varphi}|_{S} = \varphi$)

    $$
    \begin{tikzcd}
        S && G \\
        &&& {} \\
        F
        \arrow["{\overline{\varphi}}"', dashed, from=3-1, to=1-3]
        \arrow["\varphi"', from=1-1, to=1-3]
        \arrow[hook, from=1-1, to=3-1]
    \end{tikzcd}
    $$

    Grupi $F$ pravimo \textbf{prosta}, če vsebuje kakšno podmnožico, ki jo prosto generira, zgornji lastnosti pa pravimo \textbf{univerzalna lastnost} prostih grup.
\end{definicija}

\begin{izrek}[Enoličnost prostih grup]
    \label{izr:enolicnost}
    Naj bo $S$ množica. Potem do izomorfizma natančno obstaja največ ena grupa prosto generirana z $S$.
\end{izrek}

\begin{proof}
    Naj bosta $F$ in $F'$ grupi, prosto generirani z $S$. Naj bosta $\varphi : S \hookrightarrow F$ in $\varphi' : S \hookrightarrow F'$ inkluziji.
    Po univerzalni lasnosti prostih grup lahko $\varphi$ in $\varphi'$ razširimo do homomorfizmov $\overline{\varphi}$ in $\overline{\varphi'}$.

    $$
    \begin{tikzcd}
        S && {F'} \\
        &&& {} \\
        F
        \arrow["{\varphi'}", from=1-1, to=1-3]
        \arrow["\varphi"', hook, from=1-1, to=3-1]
        \arrow["{\overline{\varphi}'}"', dashed, from=3-1, to=1-3]
    \end{tikzcd}
    \hspace{40pt}
    \begin{tikzcd}
        S && F \\
        &&& {} \\
        {F'}
        \arrow["{\overline{\varphi}}"', dashed, from=3-1, to=1-3]
        \arrow["\varphi"', from=1-1, to=1-3]
        \arrow["{\varphi'}", hook, from=1-1, to=3-1]
    \end{tikzcd}
    $$

    Kompozitum homomorfizmov $\overline{\varphi} \circ \overline{\varphi}' : F \rightarrow F$ je homomorfizem, ki je na $S$ identiteta ($\overline{\varphi}$ in $\overline{\varphi}'$ na $S$ sovpadata z $\varphi$ in $\varphi'$).
    Če obravnavamo $\varphi$ kot preslikavo iz $S$ v $F$ sta torej tako $\overline{\varphi} \circ \overline{\varphi}'$ kot $id_F$ razširitvi, ki ustrezata univerzalni lastnosti in zato po enoličnosti razširitve velja $\overline{\varphi} \circ \overline{\varphi}' = id_F$.
    
    $$
    \begin{tikzcd}
        S && F \\
        &&& {} \\
        F
        \arrow["{\overline{\varphi} \circ \overline{\varphi}'}"', dashed, from=3-1, to=1-3]
        \arrow["\varphi"', from=1-1, to=1-3]
        \arrow["\varphi", hook, from=1-1, to=3-1]
    \end{tikzcd}
    \hspace{40pt}
    \begin{tikzcd}
        S && F \\
        &&& {} \\
        F
        \arrow["id_F"', dashed, from=3-1, to=1-3]
        \arrow["\varphi"', from=1-1, to=1-3]
        \arrow["\varphi", hook, from=1-1, to=3-1]
    \end{tikzcd}
    $$

    Podobno $\overline{\varphi}' \circ \overline{\varphi} = id_{F'}$, torej je $\overline{\varphi} \circ \overline{\varphi}'$ izomorfizem.
\end{proof}

\begin{izrek}[Eksistenca prostih grup]
\label{izr:freeGroupExistence}
    Naj bo $S$ množica. Potem obstaja grupa, prosto generirana z $S$.
\end{izrek}

\begin{opomba}
    Po izreku \ref{izr:enolicnost} je ta grupa enolično določena do izomorfizma natančno.
\end{opomba}

\begin{proof}
    Ideja dokaza je konstrukcija grupe sestavljene iz t.i. \textit{besed} ki so sestavljene iz elementov $S$ in njihovih inverzov, na katerih bomo uporabili le popolnoma očitno pravilo krajšanja.

    Konkretno definiramo
    $$
    A := S \cup \widehat{S}
    $$
    kot abecedo iz katere bomo sestavili naše besede. Tu $\widehat{S} = \{\widehat{s} \; | \; s \in S\}$ predstavlja disjunktno kopijo $S$ (t.j. $\widehat{\cdot} : S \longrightarrow \widehat{S}$ je bijekcija in $S \bigcap \widehat{S} = \emptyset$), ki bo v konstrukciji prevzela vlogo inverzov.

    V prvem koraku vzemimo z oznako $\As$ množico vseh končnih zaporedij iz abecede $A$. To vsebuje, med drugimi, tudi prazno besedo $\varepsilon$.
    Na $\As$ definirajmo binarno operacijo kompozicije, ki stakne skupaj dve besedi. Ta operacija je asociativna in $\varepsilon$ je nevtralni element.

    V nadaljevanju definirajmo
    $$
    F(S) := \As / \sim
    $$
    kjer je $\sim$ ekvivalenčna relacija definirana kot
    \begin{align*}
        \forall x, y \in \As \; \forall s \in S : xs\widehat{s}y \sim xy \\
        \forall x, y \in \As \; \forall s \in S : x\widehat{s}sy \sim xy
    \end{align*}
    drugače, elementa v $\As$ smatramo kot ekvivalentna, če se razlikujeta natanko za očitno uporabo pravila krajšanja (Opomba: popolnoma formalno je $\sim$ najmanjša ekvivalenčna relacija, ki zadostuje zgornjemu pogoju).
    
    Ni težko preveriti, da kompozicija besed v $\As$ inducira dobro definirano binarno operacijo $\cdot : F(S) \times F(S) \longrightarrow F(S)$ definirano kot
    $$
    [x] \cdot [y] = [xy]
    $$
    kjer so z oglatimi oklepaji označeni ekvivalenčni razredi po $\sim$.

    Pokažimo, da je množica $F(S)$ s tako operacijo grupa. Očitno je $[\varepsilon]$ nevtralni element, asociativnost pa sledi iz asociativnosti kompozicije v $\As$.
    Induktivno (po dolžini besede) definiramo preslikavo $I : \As \longrightarrow \As$ ki besedi priredi inverz kot
    \begin{align*}
        I(\varepsilon) &= \varepsilon \\
        I(sx) &:= I(x)\widehat{s} \\
        I(\widehat{s}x) &:= I(x)s
    \end{align*}
    za vse $x \in \As$ in $s \in S$. Induktivno vidimo da $I(I(x)) = x$ in
    $$
    [I(x)] \cdot [x] = [I(x)x] = [\varepsilon]
    $$
    za vse $x \in \As$ (zadnja enakost sledi iz definicije $\sim$). Zato tudi
    $$
    [x] \cdot [I(x)] = [I(I(x))] \cdot [I(x)] = [\varepsilon] \text{.}
    $$
    Torej je $F(S)$ grupa.

    Ostane name le še pokazati, da $S$ prosto generira $F(S)$.
    Naj bo $i : S \longrightarrow F(S)$ preslikava, ki vsaki črki $S \subset \As$ priredi njen ekvivalenčni razred v $F(S)$.
    Po konstrukciji $i(S)$ generira $F(S)$.
    
    Zdaj pokažimo, da ima $F(S)$ naslednjo lastnost, podobno univerzalni lastnosti prostih grup: Za vsako grupo $G$ in preslikavo $\varphi : S \longrightarrow G$
    obstaja enolično določen homomorfizem grup $\overline{\varphi} : F(S) \longrightarrow G$, da $\varphi = \overline{\varphi} \circ i$.
    $$
    \begin{tikzcd}
        S && G \\
        &&& {} \\
        {F(S)}
        \arrow["{\overline{\varphi}}"', dashed, from=3-1, to=1-3]
        \arrow["\varphi"', from=1-1, to=1-3]
        \arrow["i", from=1-1, to=3-1]
    \end{tikzcd}
    $$
    Opomnimo, da formalno gledano to ni univerzalna lastnost prostih grup, saj grupa $F(S)$ ne vsebuje množice $S$, temveč njej ustrezne ekvivalenčne razrede $i(S)$. Če je $i$ injektivna pa lahko $S$ identificiramo z $i(S)$ in to res postane univerzalna lastnost.

    Pri dokazu zgornje lastnosti z danim $\varphi$ induktivno definiramo
    \begin{align*}
        \varphi^* : \As &\longrightarrow G \\
        \varepsilon &\longmapsto e \\
        sx &\longmapsto \varphi(s) \cdot \varphi^*(x) \\
        \widehat{s}x &\longmapsto (\varphi(s))^{-1} \cdot \varphi^*(x)
    \end{align*}
    za vse $s \in S$ in vse $x \in \As$. Lahko je videti, da je $\varphi^*$ kompatibilna z ekvivalenčno relacijo $\sim$ in da $\varphi^*(xy) = \varphi^*(x) \cdot \varphi^*(y)$.
    Torej $\varphi^*$ inducira dobro definiran homomorfizem
    \begin{align*}
    \overline{\varphi} : F(S) &\longrightarrow G \\
    [x] &\longmapsto [\varphi^*(x)] \text{.}
    \end{align*}
    po konstrukciji $\varphi = \overline{\varphi} \circ i$. Ker $i(S)$ generira $F(S)$ pa je $\overline{\varphi}$ enolično določen.

    Ostane nam le še dokaz injektivnosti $i$.
    Naj bosta $s_1, s_2 \in S$ in $\varphi : S \longrightarrow \Z$ poljubna preslikava, da velja $\varphi(s_1) = 1$ in $\varphi(s_2) = -1$.
    Potem nam inducirani $\overline{\varphi}$ da
    $$
    \overline{\varphi}(i(s_1)) = \varphi(s_1) = 1 \neq -1 = \varphi(s_2) = \overline{\varphi}(i(s_2)) \text{.}
    $$
    Oziroma $i(s_1) \neq i(s_2)$.
    
    Torej je $i$ injektivna in lahko $S$ identificiramo z njegovo sliko $i(S)$, zgornja lastnost pa tako res postane univerzalna lastnost prostih grup.
\end{proof}

\begin{izrek}
    Naj bo $F$ grupa, prosto generirana z $S$. Potem je $S$ generator grupe $F$.
\end{izrek}

\begin{proof}
    Po konstrukciji trditev velja za prosto grupo $F(S)$, ki je generirana z $S$ (glej dokaz izreka \ref{izr:freeGroupExistence}). Po izreku o enoličnosti prostih gruop (glej izrek \ref{izr:enolicnost}), obstaja izomorfizem med $F(S) \cong F$, ki je na $S$ identiteta,
    iz česar sledi, da je tudi $F$ prosto generirana z $S$.
\end{proof}

Preden začnemo delati s prostimi grupami si poglejmo še alternativno konstrukcijo, ki nam bo pomagala pri razumevanju izrekov.

\begin{definicija}[Okrajšana beseda]
    Naj bo $S$ množica in $(S \cup \widehat{S})^*$ množica vseh besed nad elementi $S$ in njihovimi formalnimi inverzi.
    Naj bo $n \in \N$ in $s_1, \ldots, s_n \in S \cup \widehat{S}$. Za besedo $s_1 \ldots s_n$ pravimo, da je \textbf{okrajšana}, če velja
    $$
    s_{j+1} \neq \widehat{s}_j \quad \text{in} \quad \widehat{s}_{j+1} \neq s_j
    $$
    za vse $j \in \{1, \ldots, n-1\}$. Posebej: $\varepsilon$ je okrajšana.

    Množico vseh okrajšanih besed v $(S \cup \widehat{S})^*$ označimo z $\Freds$.
\end{definicija}

\begin{trditev}
    Naj bo $S$ množica.
    \begin{enumerate}
        \item Množica okrajšanih besed $\Freds$ nad $S \cup \widehat{S}$ tvori grupo za operacijo kompozicije definirano kot
        \begin{align*}
            \Freds \times \Freds &\longrightarrow \Freds \\
            (s_1 \ldots s_n, s_{n+1} \ldots s_m) &\longmapsto (s_1 \ldots s_{n-r} s_{n+1+r} \ldots s_{m})
        \end{align*}
        kjer so $s_1, \ldots, s_m \in S \cup \widehat{S}$ in je $r$ največje tako število, da za vsak $j \in \{0, \ldots, r-1\}$ velja
        $$
        s_{n-j} = \widehat{s}_n+1+j \quad \text{ali} \quad \widehat{s}_{n-j} = s_{n+1+j} \text{.}
        $$
        Z drugimi besedami, kompozicija je definirana s konkatinacijo dveh besed in nato okrajšavo največjega možnega števila elementov na mestu konkatinacije.
        
        \item Grupa $\Freds$ je prosto generirana z $S$.
    \end{enumerate}
\end{trditev}

\begin{proof}
    TODO
\end{proof}

\begin{posledica}
    Naj bo $S$ množica. Vsak element proste grupe $F(S) = (S \cup \widehat{S})^* / \sim$ ustreza natanko eni okrajšani besedi nad $S \cup \widehat{S}$.
\end{posledica}

\begin{proof}
    Iz izreka \ref{izr:enolicnost} sledi $F(S) \cong \Freds$.
\end{proof}

\section{Cayleyjevi grafi}

\begin{definicija}[Pot]
    \textbf{Pot} ki povezuje vozlišči $v_{0}$ in $v_{n}$ v grafu $X = (V, E)$, je zaporedje vozlišč $v_{0},\dots,v_{n} \in V$, za
    katerega velja, da je $\{v_{i}, v_{i+1}\} \in E$ za vsak $i \in \{0, \dots, n-1 \}$.
\end{definicija}

\begin{definicija}[Cayleyev graf]
    Naj bo $S$ podmnožica, ki generira grupo $G$. \textbf{Cayleyjev graf} $G$ glede na generator $S$ je graf \cay{G}{S} katerega množica vozlišč je množica $G$
    in katerega množica povezav je množica
    $$
    \{\{g, g \cdot s\}\;|\;g \in G, s \in (S \cup S^{-1}) \setminus \{e\} \},
    $$
    kjer je s $\cdot$ označeno množenje v grupi $G$.
\end{definicija}

Dve vozlišči v Cayleyjevem grafu sta si torej sosednji natanko tedaj, ko se razlikujeta le za desno množenje z elementom (ali inverzom) dane množice $S$, ki generira $G$.

\begin{izrek}
    \label{izr:freeCayleyTree}
    Naj bo $F$ grupa, prosto generirana z $S \subset F$. Potem je graf \cay{F}{S} drevo.
\end{izrek}

\begin{proof}
    Brez škode za splošnost se omejimo na $\Freds$, saj sta $F \cong \Freds$ izomorfna.
    Če imamo v grafu \cay{\Freds}{S} cikel, to pomeni, da lahko začnemo v nekem vozlišču $x$ in z zaporednim množenjem 
    z elementi $S$ (elementa sta sosednja natanko tedaj ko se razlikujeta za množenje z nekim elementom $S$) pridemo nazaj v $x$. Torej
    $$
        x s_1 \ldots s_n = x
    $$
    okrajšamo $x$ in dobimo
    $$
    s_1 \ldots s_n = \varepsilon \text{.}
    $$
    % TODO lahko imamo a b i(b) i(a) inverzi niso nujno zaporedni
    Kar je protislovno, saj je v $\Freds$ to možno natanko tedaj, ko sta si $s_i$ in $s_{i+1}$ inverzna (inverz pa pomeni potovanje po isti povezavi v obratno smer).
\end{proof}

\begin{opomba}
    Inverz v splošnem ne drži. t.j. ni res da za vsak \cay{F}{S} sledi, da je $F$ prosto generirana z $S$. TODO: protiprimer
\end{opomba}

\begin{izrek}
\label{izr:cayleyFreeCycle}
    Naj bo $G$ grupa in naj $S \subset G$ generira $G$. Dodatno naj velja, da $s \cdot t \neq e$ za vsaka $s, t \in S$.

    Če je Cayleyev graf \cay{G}{S} drevo, potem $S$ prosto generira $G$.
\end{izrek}

\begin{proof}
    Naj bo $G$ grupa in $S \subset G$ da, velja predpostavka izreka.
    Da pokažemo, da je $G$ prosto generirana, je dovolj, da pokažemo, da je $G$ izomorfna $\Freds$ z izomorfizmom, ki je na $S$ identiteta.

    Ker je $\Freds$ prosto generirana z $S$ nam univerzalna lastnost prostih grup že nudi obetavnega kandidata za izomorfizem, t.j. homomorfizem $\overline{\varphi}$, ki ga
    dobimo z razširitvijo identitete $\varphi : S \longrightarrow G$. Ker $S$ generira $G$ po predpostavki, avtomatično sledi, da je $\overline{\varphi}$ surjektiven.

    Predpostavimo da $\overline{\varphi}$ ni injektiven. Potem obstajata neka $s_1 \ldots s_n \in \Freds \setminus \{\varepsilon\}$, kjer $s_1, \ldots, s_n \in S \cup \widehat{S}$, ki se slika v enoto $\overline{\varphi}(s_1 \ldots s_n) = e$ (identificiramo sliki dveh različnih besed in obe strani množimo z inverzi črk. Ker sta besedi različni dobimo neprazno besedo ki se slika v enoto).
    Ločimo primere
    \begin{enumerate}
        \item Ker $\overline{\varphi}|_S = id_S$ mora biti $n > 1$.
        \item Če $n = 2$ potem
        $$
        e = \overline{\varphi}(s_1 s_2) = \overline{\varphi}(s_1)\overline{\varphi}(s_2) = s_1 s_2
        $$
        kar je v protislovju z našo predpostavko da $\forall  s, t \in S : s \cdot t \neq e$.
        \item Če $n \ge 3$ v \cay{G}{S} začnemo v vozlišču $e$ ter po povezavah $s_1, \ldots, s_n$ sprehodimo preko vozlišč
        \begin{align*}
            g_0 &= e \\
            g_i &= g_{i-1} s_i \quad \forall i \in \{1, \ldots, n\} \text{.}
        \end{align*}
        Ker je $s_1 \ldots s_n$ okrajšana beseda, je to zaporedje vozlišč cikel, kar pa je protislovno s predpostavko, da je \cay{G}{S} drevo.
    \end{enumerate}
\end{proof}

\section{Delovanje}

\begin{definicija}[Delovanje]
    Naj bo $G$ grupa, naj bo $C$ kategorija in naj bo $X$ objekt v $C$. \textbf{Delovanje} grupe $G$ na $X$ v kategoriji $C$ je
    homomorfizem grup $G \longrightarrow Aut_C(X)$.
    
    Z drugimi besedami, delovanje $G$ na $X$ vsakemu elementu $g \in G$ priredi ustrezni avtomorfizem $f_g : X \longrightarrow X$, da velja
    $$
    f_g \circ f_h = f_{g h}
    $$
    za vsaka $g, h \in G$.
\end{definicija}

\begin{definicija}[Prosto delovanje na množici]
    Naj grupa $G$ deluje na množici $X$. Pravimo, da je delovanje \textbf{prosto}, če velja:
    $$
        g \cdot x \neq x
    $$
    za vsak $g \in G \setminus \{e\}$ in vsak $x \in X$.
\end{definicija}

\begin{definicija}[Prosto delovanje na grafu]
    Naj grupa $G$ deluje na grafu $(V, E)$. Označimo delovanje kot preslikavo $\rho : G \longrightarrow \aut{V}{E}$. To delovanje je \textbf{prosto}, če za vsak $g \in G\setminus\{e\}$ velja:
    \begin{align*}
        \forall v \in V : (\rho(g))(v) &\neq v \text{, in}\\
        \forall \{v, v'\} \in E : \{(\rho(g))(v), (\rho(g))(v')\} &\neq \{v, v'\}
    \end{align*}

\end{definicija}

\begin{izrek}
\label{izr:cayleyFreeOrder}
    Naj bo $G$ grupa in $S$ neka množica ki generira $G$. Potem je delovanje $G$ na \cay{G}{S} z levo translacijo
    $$
    g \cdot v = gv
    $$
    prosto natanko tedaj, ko $S$ ne vsebuje nobenega elementa reda $2$.
\end{izrek}

\begin{proof}
\label{dkz:prostoDelRed}
    Ker velja $g \cdot g' = gg'$ po definiciji delovanja in $gg' = g' \Longleftrightarrow g = e$ zaradi dejstva da sta $g$ in $g'$ elementa grupe $G$,
    je delovanje $G$ na vozliščih \cay{G}{S} vedno prosto in ustreza prvemu delu definicija za prosto delovanje na grafu. Dovolj je torej, da dokažemo ekvivalentnost drugega dela definicije.
    
    Pokažimo, da v primeru, da delovanje ni prosto, \cay{G}{S} vsebuje element reda $2$. Naj bo $g \in G$ in naj bo $\{v, v'\}$ povezava v \cay{G}{S} za katerega velja $\{v, v'\} = g \cdot \{v, v'\} = \{g \cdot v, g \cdot v'\}$ (druga enakost je definicija delovanja). Iz enakosti množic ločimo dva primera:
    \begin{enumerate}
    \item Če $g \cdot v = v$ in $g \cdot v' = v'$ kar pa je res le v primeru da $g = \varepsilon$, ker je delovanje na vozliščih prosto.
    \item Če $g \cdot v = v'$ in $g \cdot v' = v$ po definiciji sosednosti v \cay{G}{S} obstaja $s \in (S \cup S^{-1})\setminus \{\varepsilon\}$ da je $v' = vs$. Sledi:
    $$
	v = g \cdot v' = g \cdot (vs) = g(vs) = (gv)s = (g \cdot v)s = (v')s = (vs)s =vs^{2}    
    $$
    Če zdaj z desne množimo z $v^{-1}$ dobimo željeno enakost $e = s^{2}$. Torej je $s$ iskani element reda $2$.
    \end{enumerate}
    
    Pokažimo še da iz tega, da v $S$ obstaja element reda $2$, sledi, da delovanje ni prosto. Naj bo $s \in S$ reda $2$. Pri delovanju $G$ na \cay{G}{S} velja 
    $$s \cdot \{e, s\} = \{s \cdot e, s \cdot s\} = \{s, s^{2}\} = \{s, e\}$$
 torej obstaja element grupe $G$ ki fiksira neko povezavo in po definiciji dano delovanje na grafu ni prosto.
\end{proof}

\begin{definicija}[Vpeto drevo delovanja]
    Naj grupa $G$ deluje na povezan graf $X$. \textbf{Vpeto drevo delovanja} $G$ na $X$ je podgraf $X$ ki je drevo in vsebuje natanko eno vozlišče vsake orbite delovanja $G$ na vozlišča grafa.
\end{definicija}

\begin{izrek}
\label{izr:vpetoDrevoDel}
    Vsako delovanje grupe na povezanem grafu ima vpeto drevo delovanja.
\end{izrek}

\begin{proof}
    Naj bo $G$ grupa ki deluje na povezanem grafu $X$. Brez škode za splošnost je $X$ neprazen, saj je drugače prazno drevo iskano vpeto drevo delovanja.
    Naj bo $T_{G}$ družina poddreves $X$ ki vsebujejo največ en element vsake orbite delovanja $G$.
    Družina $T_{G}$ je delno urejena za relacijo podgrafa, neprazna (vsebuje prazno drevo), vsaka veriga v $T_{G}$ pa ima zgornjo mejo (konkretno unijo vseh elementov verige).
    Po Zornovi lemi sledi, da $T_{G}$ vsebuje maksimalni element $T$. Ker je $X$ neprazen je tudi $T$ neprazen.

    Zdaj pokažimo, da je $T$ iskano vpeto drevo delovanja. Denimo, da $T$ ni vpeto drevo delovanja.
    Potemtakem obstaja neko vozlišče $v$, da nobeno od vozlišč v orbiti $G \cdot v$ ni vozlišče v $T$. S pomočjo vozlišča $v$ bomo poiskali vozlišče $v_{0}$ za
    katerega bo veljalo, da nobeno od vozlišč v orbiti $G \cdot v_{0}$ ni v $T$, dodatno pa ima $v_{0}$ sosedno vozlišče, ki je v grafu $T$, iz česar bo sledilo protislovje.

    Ker je $X$ povezan obstaja pot $p$, ki povezuje neko vozlišče $u \in T$ z $v$. Naj bo $v'$ prvo vozlišče v poti $p$, da $v' \notin T$. Ločimo dva primera:
    \begin{enumerate}
        \item Nobeno izmed vozlišč v $G \cdot v'$ ni v $T$. Potem je to iskano vozlišče $v_{0} := v'$.
        \item Obstaja $g \in G$, da $g \cdot v' \in T$. Označimo s $p'$ pot med $v'$ in $v$, ter z $g \cdot p'$ pot med $g \cdot v'$ in $g \cdot v$, kjer smo
              vsako vozlišče poti $p'$ `premaknili` z delovanjem $g$. Ker je tako premaknjen $g \cdot v' \in T$, pot $g \cdot p'$ pa krajša od poti $p$, lahko
              postopek induktivno nadaljujemo, dokler ne najdemo želenega vozlišča (vozlišče zagotovo najdemo, ker za $v$ noben element $G \cdot g \cdot v = G \cdot v$ ni v $T$).
    \end{enumerate}

    Naj bo zdaj $v$ vozlišče za katerega noben element $G \cdot v$ ni v $T$ in ima soseda $u \in T$. Če zdaj drevesu $T$ dodamo vozlišče $v$ in povezavo $\{u, v\}$
    dobimo drevo v $T_{G}$, ki vsebuje $T$ kot pravo poddrevo, kar je skregano z dejstvom, da je $T$ maksimalno. Sledi, da je $T$ iskano vpeto drevo delovanja grupe $G$ na povezanem grafu.
\end{proof}

\section{Delovanje prostih grup na drevesih}

\begin{izrek}
    Grupa je prosta natanko tedaj ko ima neko prosto delovanje na nepraznem drevesu.
\end{izrek}

\begin{proof}
    $(\Longrightarrow)$

    Naj bo $F$ prosta grupa prosto generirana z $S \subset F$ po izreku \ref{izr:freeCayleyTree} je njen Cayleyev graf \cay{F}{S} (neprazno) drevo. Po izreku \ref{izr:cayleyFreeOrder} je delovanje $F$ na \cay{F}{S} z levo translacijo prosto natanko tedaj ko v $S$ ni elementa reda $2$. Uporabimo univerzalno lastnost prostih grup, da se prepričamo da $S$ res nima elementov reda $2$.
    
    Naj bo $\varphi : S \longrightarrow \Z$ poljubna preslikava, za katero velja, da noben $s \in S$ ne slika v $0$. Po univerzalni lastnosti jo lahko dopolnemo do homomorfizma $\overline{\varphi} : F \longrightarrow \Z$. Če $s \in S$ reda $2$ mora red elementa $\overline{\varphi}(s)$ deliti $2$. Red elementa $\overline{\varphi}(s)$ 	ne more biti $2$, saj $(\Z, +)$ ne vsebuje elementov reda 2. Hkrati pa red elementa $\overline\varphi(s)$ ne more biti $1$, ker je v $\Z$ edini element reda $1$ element $0$, ki pa po izbiri $\varphi$ ni slika nobenega $s \in S$. (Spomnimo se, da $\overline\varphi|_{S}$ sovpada s $\varphi$)
    
    Torej je delovanje proste grupe $F$ na drevesu \cay{F}{S} z levo translacijo prosto.

    $(\Longleftarrow)$
    
    Naj ima grupa $G$ neko prosto delovanje na drevesu $T$. Po izreku \ref{izr:vpetoDrevoDel} za to delovanje obstaja vpeto drevo delovanja.

    Ideja dokaza je da znotraj grafa $T$ kontraktiramo $T'$ in vsako njegovo translacijo $g \cdot T'$ (kjer $g \in G$) vsako v eno samo vozlišče.
    S tem bomo dobili kandidata za množico $S$. Da $S$ res prosto generira $G$ pa bo sledilo po izreku \ref{izr:cayleyFreeCycle}.

    Preden začnemo s konstrukcijo kandidata poimenujmo z besedno zvezo \textbf{esencialne povezave}, povezave, v drevesu $T$, ki niso vsebovane v $T'$,
    eno izmed vozlišč med katerima potekajo, pa je vsebovano v $T'$
    (drugo vozlišče ni vsebovano v $T'$, saj bi drugače $T$ vseboval cikel).

    Začnimo zdaj s \textit{konstrukcijo kandidata} $S \subset G$. Naj bo $e = \{u, v\}$ esencialna povezava $T$. Brez škode za splošnost je $u \in T'$ in $v \notin T'$.
    Ker je $T'$ vpeto drevo delovanja, obstaja $g_{e} \in G$, da $g_{e}^{-1} \cdot v \in T'$.
    Dodatno, ker orbita $G \cdot v$ vsebuje natanko en element v $T'$ in ker $G$ prosto deluje na $T$ sledi, da je $g_{e}$ enolično določen.

    Definirajmo
    $$
    \St := \{g_e \in G\;|\;e\;\text{je esencialna povezava}\;T \}\text{.}
    $$

    Za množico $\St$ velja:
    \begin{enumerate}
        \item Po konstrukciji enota ni vsebovana v $\St$.
        \item $\St$ ne vsebuje elementov reda $2$.
        
        Če $g_e \in \St$ reda $2$ za esencialno povezavo $e = \{u, v\}$ kot zgoraj, sledi
        $u_0 := g_e^{-1} \cdot v = g_e \cdot v \in T'$. Povezavo $e$ slikamo v $g_e \cdot e = \{g_e \cdot u, u_0\}$
        \begin{enumerate}
            \item $u = u_0 \Rightarrow g_e \cdot e = e$ protislovje s predpostavko o prostem delovanju.
            \item $u \neq u_0 \Rightarrow \{u_0, g_e \cdot u\} \in g_e \cdot T = T$ (delovanje $g_e$ je automorfizem).
            Znotraj $T'$ obstaja pot med $u_0$ in $u$, znotraj $g_e \cdot T'$ (disjunkten s $T'$) pa pot med $g_e \cdot u$ in $v$.
            Ker sta $e$ in $g_e \cdot e$ povezavi $T$ imamo v $T$ cikel, kar je protislovje s tem, da je $T$ drevo.
        \end{enumerate}

        \item Če sta $e$ in $e'$ esencialni povezavi za kateri sta $g_e = g_{e'}$, potem $e = e'$ ($T$ je drevo, zato, kot zgoraj, ne moreta obstajati dve različni povezavi med povezanima $T'$ in $g_e \cdot T' = g_{e'} \cdot T'$).
        \item Če je $g \in \St$, denimo $g = g_e$ za neko esencialno povezavo $e$, potem je tudi $g^{-1} \cdot e$ esencialna povezava in $g^{-1} = g_{g^{-1} \cdot e} \in \St$.
        
        Z drugimi besedami, obstaja podmnožica $S \subset \St$, da velja:
        \begin{align*}
            S \cap S^{-1} = \emptyset && \text{in} && |S| = \frac{|\St|}{2} = \frac{1}{2} \cdot \#\;\text{esencialnih povezav}\;T\text{.}
        \end{align*}
    \end{enumerate}

    V naslednjem koraku pokažimo, da $\St$ (in posledično $S$) \textit{generira} $G$.
    Naj bo $g \in G$ poljuben in $v \in T'$ poljubno vozlišče. Ker je $T$ povezan v njem med vozliščema $v$ in $g \cdot v$ obstaja pot $p$.
    Pot $p$ gre skozi več kopij $T'$. Označimo s $g_0 \cdot T', \ldots , g_n \cdot T'$ zaporedne kopije $T'$ skozi katere
    vodi $p$, tako da velja $g_0 = e$, $g_n = g$ in $g_i \neq g_{i+1}$ za vsak $i \in \{0, \ldots, n-1\}$.
    Naj bo zdaj $j \in \{0, \ldots, n-1\}$ poljuben. Ker je $T'$ vpeto drevo delovanja in je $g_i \neq g_{i+1}$, sta kopiji
    $g_j \cdot T'$ in $g_{j+1} \cdot T'$ povezani z neko povezavo $e_j$. Po definiciji je $g_j^{-1} \cdot e_j$ esencialna povezava kateri smo v $\St$ priredili element
    $$s_j := g_j^{-1}g_{j+1}$$
    (bralec naj se sam prepriča da velja $s_j^{-1} \cdot v \in T'$, kjer $v \in g_j^{-1} \cdot e_j$, $v \notin T'$).

    Vidimo, da velja:
    \begin{align*}
        g &= g_n = g_0^{-1} g_n \\
          &= g_0^{-1} g_1 g_1^{-1} g_2 \dotsm g_{n-1}^{-1} g_n \\
          &= s_0 s_1 \dotsm s_{n-1}
    \end{align*}
    in $g$ je element podgrupe generirane s $\St$. Ker je bil $g$ poljuben sledi, da $\St$ generira $G$.
    Vidimo, da lahko, ko znotraj $T$ kontraktiramo vse kopije $T'$ vsako v eno samo vozlišče, novo dobljeni graf gledamo kot \cay{G}{\St}.

    Ostane nam le še premislek, da $S \subset \St$, ki smo ga definirali zgoraj \textit{prosto generira} $G$. Po izreku \ref{izr:cayleyFreeCycle} zadošča dokazati, da \cay{G}{S} ne vsebuje ciklov (predpostavki $\forall s,t \in S : s \cdot t \neq e$ smo zadostili s $4$. lastnostjo $\St$),
    kar pa bo držalo, ker lahko v primeru da imamo cikel, \cay{G}{S} razširimo nazaj v $T$ ter pridemo v protislovje s predpostavko, da je $T$ drevo.
    
    Naredimo torej to. Denimo, da obstaja $n \in \N$, $n \ge 3$ za katerega v \cay{G}{S} $=$ \cay{G}{\St} obstaja cikel $g_0, \ldots, g_{n-1}$. Kot zgoraj so
    \begin{align*}
        s_{j+1} &:= g_j^{-1}g_{j+1} \quad \forall j \in \{0, 1, \ldots, n-2\} \\
        s_n &:= g_{n-1}^{-1}g_0
    \end{align*}
    elementi $\St$. Naj bo $e_i$ esencialna povezava med $T'$ in $s_i \cdot T'$ za $i \in \{1, \ldots, n\}$.
    Oglejmo si povezavi $e_i$ in $e_{i+1}$ za $i \in \{1, \ldots, n-2\}$. Njuni translaciji $g_{i-1} \cdot e_i$ in $g_{i} \cdot e_{i+1}$ povezujeta translaciji $g_{i-1} \cdot T'$ z $g_{i} \cdot T'$ ter $g_{i} \cdot T'$ z $g_{i+1} \cdot T'$.
    Ker je $g_i \cdot T'$ drevo, znotraj njega obstaja pot med tistim vozliščem povezav $g_{i-1} \cdot e_i$ in $g_{i} \cdot e_{i+1}$ ki je v $g_{i} \cdot T'$.
    Vidimo torej, da obstaja pot med $g_0 \cdot T'$ in $g_{n-1} \cdot T'$. Ker pa $g_{n-1} \cdot e_n$ povezuje $g_{n-1} \cdot T'$ in $g_0 \cdot T'$ pomeni, da v $T$ obstaja cikel, s čimer smo prišli v iskano protislovje.

\end{proof}
    
\section{Nielsen-Schreirerjev izrek}
	
\end{document}