\documentclass{beamer}

\usepackage[slovene]{babel}
\usepackage{amsfonts,amssymb}
\usepackage[utf8]{inputenc}
\usepackage{lmodern}
\usepackage[T1]{fontenc}

\usetheme{Warsaw}

\def\N{\mathbb{N}} % mnozica naravnih stevil
\def\Z{\mathbb{Z}} % mnozica celih stevil
\def\Q{\mathbb{Q}} % mnozica racionalnih stevil
\def\R{\mathbb{R}} % mnozica realnih stevil
\def\C{\mathbb{C}} % mnozica kompleksnih stevil

\newcommand{\cay}[2]{$Cay(#1, #2)$}
\newcommand{\aut}[2]{Aut(#1, #2)}
\newcommand{\cit}[1]{(Izrek \ref{#1})}
\newcommand{\St}{\widetilde{S}}
\newcommand{\As}{A^*}
\newcommand{\Freds}{F_{red}(S)}


\def\qed{$\hfill\Box$}   % konec dokaza
\newtheorem{izrek}{Izrek}
\newtheorem{trditev}{Trditev}
\newtheorem{posledica}{Posledica}
\newtheorem{lema}{Lema}
\newtheorem{definicija}{Definicija}
\newtheorem{opomba}{Opomba}
\newtheorem{primer}{Primer}
\newtheorem{zgled}{Zgled}
\newtheorem{zgledi}{Zgledi uporabe}
\newtheorem{zglediaf}{Zgledi aritmetičnih funkcij}
\newtheorem{oznaka}{Oznaka}

\institute{Fakulteta za matematiko in fiziko \\
Oddelek za matematiko}

\title{Proste grupe in drevesa}
\author{Jakob Pogačnik Souvent}

\begin{document}


%%%%%%%%%%%%%%%%%%%%%%%%%%%%%%%%%%%%%%%%%%%%%%%%%%%%%%%%%%%%%%%%%%%%%

\begin{frame}
\titlepage
\end{frame}

%%%%%%%%%%%%%%%%%%%%%%%%%%%%%%%%%%%%%%%%%%%%%%%%%%%%%%%%%%%%%%%%%%%%%
% Uvod, definicija prostih grup, motivacija (zakaj so nam všeč), izrek do katerega bomo delali (ukvarjali se bomo s prepoznavanjem pr. gr.)

\begin{frame}

    \frametitle{Uvod}

    \pause

    % Preden začnemo z grupami se najprej ustavimo pri vektorskih prostorih.
    % Ponavadi imamo radi da imajo vektorji v naših generatorjih eno zelo lepo lastnost.
    % Kaj bi bila ta lastnost za vektorske prostore? (Če čakajo: razlika med linearno ogrinjačo in bazo, obe generirata, vendar ena ima to lastnost ena pa ne)    
    % 
    % Najbolj "prost" generator vektorskih prostorov je torej tak ki ima same linearno neodvisne vektorje
    % kako pa zdaj naredimo analogijo za grupe?
    % Uvedimo definicijo.

    \begin{definicija}[Proste grupe]
    %TODO: Style
        Naj bo $S$ množica. Za grupo $F$, ki vsebuje $S$ Pravimo, da $S$ \textbf{prosto generira} $F$, če velja:

        Za vsako grupo $G$ in vsako preslikavo $\varphi : S \longrightarrow G$ obstaja enolično določen homomorfizem $\overline{\varphi} : F \longrightarrow G$, ki razširi $\varphi$.
    \end{definicija}

    \pause
    
    % POVEJ:
    % Grupi $F$ pravimo \textbf{prosta}, če vsebuje kakšno podmnožico, ki jo prosto generira, zgornji lastnosti pa pravimo \textbf{univerzalna lastnost} prostih grup.

    % NARIŠI:
    % Comutativni diagram

    % Oglejmo si nekaj primerov proste grupe

    % Razloži zakaj.
    % Preslikave v Z_2
    \begin{zgled}
        \begin{enumerate}
            \item Trivialna grupa je prosto generirana s prazno množico.
            \item $\Z$ je prosto generirana z $\{1\}$, vendar ne z $\{2, 3\}$ ali z $\{2\}$.
            \item $\Z_2$ ni prosto generirana.
        \end{enumerate}
    \end{zgled}

    % Opazimo, da je težje pokazati da grupa ni prosto generirana.
\end{frame}

\begin{frame}
    \frametitle{Motivacija}
    \pause
    % Motivacijski slajd

    % Preden bi nadaljevali pa se vprašajmo, zakaj nas sploh zanimajo grupe s tako lastnostjo?
    % Če naredimo korak nazaj zakaj nas sploh zanimajo algebraične strukture s prostimi generatorji?
    %
    % Odgovor:
    % Ponavadi je v algebri vsaka struktura kvocient t.i. proste strukture.
    % In za grupe velja naslednji izrek

    \begin{izrek}
        Vsaka grupa je kvocientna grupa neke proste grupe.
    \end{izrek}
    % Če torej razumemo proste grupe nam to pripomore k razumevanju vseh grup
    \pause
    % V današnjem predavanju se bomo osredotočili predvsem na razumevanje povezav med grupami in grafi v cilju prepoznavanja prostih grup
    % in da bomo imeli med predavanjem v mislih nek cilj kar zdaj razkrijemo izrek 
    \begin{izrek}
        Grupa je prosta natanko tedaj, ko ima neko prosto delovanje na nepraznem drevesu.
    \end{izrek}
    
    % preden se lotimo dokaza te trditve pa vidimo da nas čaka še nekaj definicij izrekov
\end{frame}

%%%%%%%%%%%%%%%%%%%%%%%%%%%%%%%%%%%%%%%%%%%%%%%%%%%%%%%%%%%%%%%%%%%%%
% Brez dokaza nekaj pomembnih elementarnih lasnosti
% Pr. gr. enolicnost, eksistenca, generator, konstrukcija freds

\begin{frame}
    \frametitle{Osnovne lastnosti}

    \pause

    % Začnimo najprej z nekaj elementarnimi lastnostmi prostih grup, ki jih bomo povedali brez dokaza
    % če koga zanimajo dokazi si jih lahko ogleda v člankih.
    
    % Lastnosti omenimo zato ker se bomo nanje nanašali v nekaterih kasnejših dokazih.

    \begin{izrek}[Enoličnost prostih grup]
        Naj bo $S$ množica. Potem do izomorfizma natančno obstaja največ ena grupa prosto generirana z $S$.
        % Povej idejo dokaza. Trivialen
    \end{izrek}
    \pause
    \begin{izrek}[Eksistenca prostih grup]
        Naj bo $S$ množica. Potem obstaja grupa, prosto generirana z $S$.
        % Povej idejo dokaza. Zahteven s konstrukcijo
    \end{izrek}
    \pause
    % Zakaj se ne ustavljamo dolgo na tej konstrukciji prostih grup je zato ker obstaja lepša konstrukcija
    % t.j. konstrukcija z okrajšanimi besedami, ki je po enoličnosti ekvivalentna.

    % Tukaj je vredno opomniti da to, da S generira F ni del definicije
    \begin{izrek}
        Naj bo $F$ grupa, prosto generirana z $S$. Potem je $S$ generator grupe $F$.
    \end{izrek}

\end{frame}

\begin{frame}
    \frametitle{Konstrukcija}
    \begin{trditev}
        Naj bo $S$ množica.
        \begin{enumerate}
            \item Množica okrajšanih besed $\Freds$ nad $S \cup \widehat{S}$ tvori grupo za operacijo kompozicije
            \item Grupa $\Freds$ je prosto generirana z $S$.
        \end{enumerate} 
    \end{trditev}

    % razloži kaj je abeceda
    % kaj je beseda
    % kaj pomeni biti okrajšan

    % no in če tako definiramo operacijo zdaj imamo neko konkretno presentacijo
    % proste grupe. zakaj nam je to všeč. spomnemo se na prejšnji izrek o enoličnosti
    % če poznamo eno prosto grupo poznamo vse

    % dejstvo da nam taka konstrukcija da prosto grupo bo pomembno v naslednjih dokazih

    % ZGLED: izmisli si množico, {a, b, c, d, e, f} napiši par besed in naredi operacijo

\end{frame}

%%%%%%%%%%%%%%%%%%%%%%%%%%%%%%%%%%%%%%%%%%%%%%%%%%%%%%%%%%%%%%%%%%%%%
% Preidimo na grafe. Ogledli si bomo poseben graf ki je povezan z grupami

% Def cayley
% Kako je povezan s prostimi grupami
% cayley is tree
% tree is free

\begin{frame}
    \frametitle{Grafi}
    \pause
    \begin{definicija}[Cayleyev graf]
        Naj bo $S$ podmnožica, ki generira grupo $G$. \textbf{Cayleyjev graf} $G$ glede na množico generatorjev $S$ je graf \cay{G}{S}, katerega množica vozlišč je množica $G$
        in katerega množica povezav je množica
        $$
        \{\{g, g \cdot s\}\;|\;g \in G, s \in (S \cup S^{-1}) \setminus \{e\} \},
        $$
        kjer je s $\cdot$ označeno množenje v grupi $G$.
    \end{definicija}
    \pause
    \begin{zgled}
        \begin{enumerate}
            \item \cay{\Z}{\{1\}}
            \item \cay{\Z}{\{2,3\}}
            \item \cay{\Z^2}{\{(1,0), (0,1)\}}
            \item \cay{\Z_6}{\{1\}}
        \end{enumerate}
    \end{zgled}
\end{frame}

\begin{frame}
    \frametitle{Cayleyev graf prostih grup}
    \begin{zgled}
        \cay{\Freds}{S} kjer $S = \{a, b\}$
    \end{zgled}
    \pause
    \begin{izrek}[Cayleyev graf prostih grup]
        Naj bo $F$ grupa, prosto generirana z množico $S \subset F$. Potem je graf \cay{F}{S} drevo (drevo je graf brez ciklov).
    \end{izrek}
    \pause
    \begin{opomba}
        Obrat v splošnem ne velja. Protiprimer: \cay{\Z_2}{\{1\}}.
    \end{opomba}

\end{frame}

\begin{frame}
    \frametitle{Cayleyev graf prostih grup}
    \begin{izrek}[Obrat]
        Naj bo $G$ grupa in naj $S \subset G$ generira $G$. Dodatno naj velja:
        
        $$\forall s, t \in S : s \cdot t \neq e \text{.} $$

        Če je Cayleyev graf \cay{G}{S} drevo, potem $S$ prosto generira $G$.    
    \end{izrek}
\end{frame}

%%%%%%%%%%%%%%%%%%%%%%%%%%%%%%%%%%%%%%%%%%%%%%%%%%%%%%%%%%%%%%%%%%%%%
% Vemo že skoraj vse pojme v izreku, manjka nam še povezava med grupo in drugo strukturo
% Def delovanje, prosto del
% leva translacija prosta iff brez reda 2

\begin{frame}
    \frametitle{(Prosto) delovanje}
    \begin{definicija}[Delovanje na grafu]
        Naj bo $G$ grupa. Naj bo $X$ graf. \textbf{Delovanje} grupe $G$ na grafu $X$ je
        homomorfizem grup $G \longrightarrow Aut(X)$.
    \end{definicija}
    \pause
    \begin{definicija}[Prosto delovanje na grafu]
        Naj bo $\rho : G \longrightarrow Aut(X)$ delovanje grupe $G$ na grafu $X = (V, E)$. Pravimo, da je to delovanje \textbf{prosto}, če velja:
        \begin{align*}
            \forall v \in V : (\rho(g))(v) &\neq v \text{, in}\\
            \forall \{v, v'\} \in E : \{(\rho(g))(v), (\rho(g))(v')\} &\neq \{v, v'\}
        \end{align*}
        za vsak $g \in G\setminus\{e\}$.
    
    \end{definicija}
\end{frame}

\begin{frame}
    \frametitle{Prosto delovanje}

    \begin{zgled}
        \begin{enumerate}
            \item Delovanje $\Z$ na $\{z \in \C\;|\;|z| = 1\}$ kot
            $$
            n \cdot z = e^{2 \pi i \alpha n} z
            $$
            kjer $\alpha \in \R$.
            \item Delovanje $\Z$ na \cay{\Z}{\{1\}} z (levim) seštevanjem.
        \end{enumerate}
    \end{zgled}
\end{frame}

\begin{frame}
    \frametitle{Prosto delovanje}

    \begin{izrek}
            Naj bo $G$ grupa in $S$ neka množica, ki generira $G$. Potem je delovanje $G$ na \cay{G}{S} z levo translacijo
            $$
            g \cdot v = gv
            $$
            prosto natanko tedaj, ko $S$ ne vsebuje nobenega elementa reda $2$.
        \end{izrek}
\end{frame}

%%%%%%%%%%%%%%%%%%%%%%%%%%%%%%%%%%%%%%%%%%%%%%%%%%%%%%%%%%%%%%%%%%%%%
% Preden lahko dokažemo manjka še en manjši pojem in izrek
% Def vpeto drevo
% vsako delovanje ima vpeto drevo

\begin{frame}
    \frametitle{Vpeto drevo}

    \begin{definicija}[Vpeto drevo delovanja]
        Naj grupa $G$ deluje na povezanem graf $X$. \textbf{Vpeto drevo delovanja} $G$ na $X$ je podgraf grafa $X$, ki je drevo in vsebuje natanko eno vozlišče vsake orbite delovanja $G$ na vozlišča grafa.
    \end{definicija}

    \begin{izrek}
        Vsako delovanje grupe na povezanem grafu ima vpeto drevo delovanja.
    \end{izrek}
\end{frame}

%%%%%%%%%%%%%%%%%%%%%%%%%%%%%%%%%%%%%%%%%%%%%%%%%%%%%%%%%%%%%%%%%%%%%

\begin{frame}
    \frametitle{Prosto delovanje na drevesu}

    \begin{izrek}
        Grupa je prosta natanko tedaj, ko ima neko prosto delovanje na nepraznem drevesu.
    \end{izrek}
\end{frame}

\end{document}